% =========================================================================== %

\begin{frame}[t,plain]
\titlepage
\end{frame}

% =========================================================================== %

\begin{frame}{Recap}
%
\begin{columns}[T]
\column{.5\linewidth}
\begin{itemize}
\item Machine Learning: Make a computer find solutions on its own
\item Several strategies: decision trees, support vector machines, ...
\item Making use of the strenghts of computers: arithmetics
\item In Detail: Deep Learning
\item Implementation with keras frontend
\end{itemize}
%
\column{.5\linewidth}
\begin{itemize}
\item Representation of input data as vector
\item Apply knowledge: Matrix multiplication
\item Learning: Changing matrix elements
\item Cost function and method of steepest descent
\item Activation functions and other layer types
\end{itemize}
%
\end{columns}
%
\begin{center}
	\emph{Any Questions?}
\end{center}
%
\end{frame}

% =========================================================================== %

\begin{frame}{Git Commit}
%
\begin{center}
	\includegraphics[width=.5\linewidth]{./gfx/xkcd-gitcommit}
	
	\vspace{12pt}
	\emph{Merge branch 'asdfasjkfdlas/alkdjf' into 'sdkjfls-final'}
	
	Source: \url{https://xkcd.com/1296/}
\end{center}
%
\end{frame}

% =========================================================================== %

\begin{frame}{Maintaining Big Projects}
%
As our projects grow bigger, we face new problems:
\begin{columns}[T]
\column{.3\linewidth}
\begin{itemize}
\item Overviewing the outline
	\begin{itemize}
	\item When working in a team
	\item Even when working alone
	\end{itemize}
\item[\Thus] Need Documentation!
\end{itemize}
%
\column{.3\linewidth}
\begin{itemize}
\item Keeping the routines bug-free
	\begin{itemize}
	\item Routines on their own
	\item Interactions with other parts
	\end{itemize}
\item[\Thus] Need Testing!
\end{itemize}
%
\column{.3\linewidth}
\begin{itemize}
\item Recovering from accidents
	\begin{itemize}
	\item Hardware failure
	\item Incompatible evolutions
	\item Being stupid
	\end{itemize}
\item[\Thus] Need Backups!
\end{itemize}
\end{columns}
%

\vspace{12pt}
Of course, we could do all of that manually, but... \emph{nah!}
%
\end{frame}

% =========================================================================== %

\begin{frame}[fragile]{DocStrings and Nametags}
%
\begin{itemize}
\item Modules, funcitons and classes have two Dunders: \inPy{__name__} and \inPy{__doc__}
\item \inPy{__name__}:
	\begin{itemize}
	\item Usually: File name without extension (\texttt{myModule.py} \thus \texttt{myModule})
	\item Exception: if not imported: \inPy{__main__}
	\end{itemize}
\item \inPy{__doc__}:
	\begin{itemize}
	\item Arbitrary text
	\item Text between triple double quotes: \inPy{"""this is a docstring"""}
	\item Can be multiline
	\item Will be shown when invoking \inPy{help}
	\item Can be used with other tools!
	\item Templates by Spyder
	\end{itemize}
\end{itemize}
%
\end{frame}

% =========================================================================== %

\begin{frame}[fragile]
%
\begin{codebox}[Example: Function DocString]
\begin{minted}[linenos, fontsize=\scriptsize]{python3}
def printHead(text, width = 80) :
    """
    Prints a headline box with text <text> of <width> characters in width.
    
    Example:
    >>> printHead('headline', 40)
    ########################################
    #               headline               #
    ########################################
    """
    
    formattext = "{" + f":^{width - 4}" + "}"            # e.g. "{:^36}"
    
    print("#" * width)
    print("# " + formattext.format(text) + " #")
    print("#" * width)

help(printHead)
\end{minted}
\end{codebox}
%
\end{frame}

% =========================================================================== %

\begin{frame}[fragile]
%
\begin{codebox}[Example: Module DocString: TextTools.py]
\begin{minted}[linenos, fontsize=\scriptsize]{python3}
""" 
Text Tools

This module offers helper tools for putting nice text on terminal outputs
"""

def printHead(text, width = 80) :
    ...

if __name__ = '__main__' :
    ...
    # your test code here!
\end{minted}
\end{codebox}
%
\begin{codebox}[Example: Module Docstring: TextTools.py]
\begin{minted}[linenos, fontsize=\scriptsize]{python3}
import TextTools

help(TextTools)
\end{minted}
\end{codebox}
%
\end{frame}

% =========================================================================== %

\begin{frame}[fragile]
%
\begin{tcbraster}[raster columns=2,
                  raster equal height,
                  nobeforeafter,
                  raster column skip=0.5cm]
\begin{codebox}[Example: Class DocString]
\begin{minted}[fontsize=\scriptsize, linenos]{python3}
class Master :
    """ A masterfully created
        piece of code! """
    
    def __init__(self) :
        """ instantiates a new
            piece of baffling
            awesome software """
        print("master created")
    
    ...
\end{minted}
\end{codebox}
%
\begin{cmdbox}[Output: \texttt{help(Master)}]
\begin{minted}[fontsize=\scriptsize, linenos]{text}
class Master(builtins.object)
 |  A masterfully created piece of code!
 |  
 |  Methods defined here:
 |  
 |  __init__(self)
 |      instantiates a new piece of
 |      baffling awesome software
 |  
 |  --------------------------------------
 |  Data descriptors defined here:
 |  
 |  __dict__
 |      dictionary for instance variables
 |     (if defined)
 |  
 |  __weakref__
 |      list of weak references to the
 |      object (if defined)
\end{minted}
\end{cmdbox}
\end{tcbraster}
%
\end{frame}

% =========================================================================== %

\begin{frame}{How Is This Better Than Comments?}
%
\begin{itemize}
\item Available \enquote{on demand} (without browsing the source)
\item Allows using other features (see below)
\item Two methods for inserting comments allow classification
	\begin{itemize}
	\item \enquote{Normal} comments (\texttt{\#}): explain how code \emph{works}
	\item[\Thus] Needed when you want to \emph{edit} the code
	\item Docstrings: explain how code \emph{is used}
	\item[\Thus] Needed when you want to \emph{use} the code
	\end{itemize}
\end{itemize}
%
\end{frame}

% =========================================================================== %

\begin{frame}[fragile]
%
\begin{codebox}[Example: DocStrings and Comments]
\begin{minted}[linenos, fontsize=\scriptsize]{python3}
import numpy as np
def numpy_isqrt(number):
    """ approximates 1 / sqrt(number) FAST """
    
    # source: https://github.com/ajcr/ajcr.github.io/blob/
    #              master/_posts/2016-04-01-fast-inverse-square-root-python.md
    # details: https://en.wikipedia.org/wiki/Fast_inverse_square_root#Algorithm
    
    y = np.float32(number)                        # ensure y is a single precision float    
    i = y.view(np.int32)                          # typecast float -> int
    i = np.int32(0x5f3759df) - np.int32(i >> 1)   # evil bit magic, making use of:
                                                  #   log_2(1/sqrt(x)) = -1/2 log_2(x)
    y = i.view(np.float32)                        # typecast int -> float
    
    # one iteration of Newton's method to increase accuracy
    threehalfs = 1.5
    x2 = number * 0.5
    y = y * (threehalfs - (x2 * y * y))
    return float(y)
\end{minted}
\end{codebox}
%
\end{frame}

% =========================================================================== %

\begin{frame}{External Documents}
%
\begin{itemize}
\item Users prefer not to search through (extensive) source codes
\item Instead: External PDF/HTML/... documentation
\item A lot of work to write and maintain!
\item doxygen: tool to \emph{gen}erate professional looking \emph{docs} (hence the name)
\item Uses DocStrings and Comments to compile a documentation
\item Compatible with Python, C, C++, Java, Fortran, VHDL and other languages
\item Command line driven; graphical front doxywizard end exists
\item Details: \url{https://www.doxygen.nl/manual/index.html}
\end{itemize}
%
\end{frame}

% =========================================================================== %

\begin{frame}
%
\begin{hintbox}[Docs Are Not Just For Other People!]
Never underestimate the positive impact of documenting your own code!

\vspace{4pt}
First of all, you'll look up frequently once projects get bigger. Having everything in a uniform and indexed and hyperlinked format saves a lot of time and avoids bugs. You'll notice the difference soon enough!

\vspace{4pt}
Second, writing a good documentation forces you to clearly specify what your classes, methods, ... do. The mere act of clearly defining interfaces and effects removes bugs before they are written!

\vspace{4pt}
To make full use of the mental benefits, write the documentation \emph{before} you write the code!
\end{hintbox}
%
\end{frame}

% =========================================================================== %

\begin{frame}{Type Hints}
%
\begin{itemize}
\item Python is \emph{dynamically typed} (\enquote{ducktyped})
	\begin{itemize}
	\item Data type of objects can change at runtime
	\item No type checking by the system
	\item Can lead to problems, \eg \inPy{math.sqrt("nine")}
	\end{itemize}
\item Type hints: Telling Python what to expect
	\begin{itemize}
	\item In parameter list: \inPy{parameter : type}
	\item Return type: \inPy{(...) -> type}
	\item Result: Dunder \inPy{__annotations__} has a \inPy{dict} of form \inPy{{'parameter' : type}}
	\item Return type in this \inPy{dict} with key \inPy{'return'}
	\end{itemize}
\item Type hints ignored by Python!
	\begin{itemize}
	\item But: Helps you as a develloper to remember the correct use
	\item External tools can use this!
	\item \enquote{Code is more often read than written.}
		\begin{flushright}
		\emph{Guido van Rossum, benevolent dictator for life}
		\end{flushright}
	\end{itemize}
\end{itemize}
%
\end{frame}

% =========================================================================== %

\begin{frame}[fragile]
%
\begin{codebox}[Example: Type Hints]
\begin{minted}[linenos, fontsize=\scriptsize]{python3}
import numpy as np
def numpy_isqrt(number : float ) -> float :
    i = np.float32(number).view(np.int32)
    i = np.int32(0x5f3759df) - np.int32(i >> 1)
    y = i.view(np.float32)
    x2 = number * 0.5
    y = y * (1.5 - (x2 * y * y))
    return float(y)

print( numpy_isqrt.__annotations__ )
print( numpy_isqrt(9) )
print( numpy_isqrt("9") )
\end{minted}
\end{codebox}
%
\begin{cmdbox}[Output: Type Hints]
\begin{minted}[linenos, fontsize=\scriptsize]{text}
{'number': <class 'numpy.float32'>, 'return': <class 'float'>}
0.3329532012819704
TypeError: can't multiply sequence by non-int of type 'float'
\end{minted}
\end{cmdbox}
%
\end{frame}

% =========================================================================== %

\begin{frame}[fragile]{Type Hints: So What Are They Good For?}
%
\vspace{-15pt}
\begin{columns}[t]
\column{.6\linewidth}
\begin{itemize}
\item Spyder: Use for generation of DocStrings
\item MyPy: External Tool
	\begin{itemize}
	\item Test all calls
	\item Written in Python
	\end{itemize}
\item Install
	\begin{itemize}
	\item \texttt{conda install -c conda-forge mypy}
	\item Or: \texttt{pip3 install mypy}
	\item Or \texttt{sudo apt install mypy} (Linux)
	\end{itemize}
\item Use
	\begin{itemize}
	\item \texttt{python3 -m mypy yourcode.py}
	\item \texttt{mypy yourcode.py}
	\end{itemize}
\end{itemize}
%
\column{.45\linewidth}
\begin{codebox}[Example: Auto-DocString]
\begin{minted}[linenos, fontsize=\scriptsize]{python3}
def someFunc (param : int) -> str :
    """


    Parameters
    ----------
    param : int
        DESCRIPTION.

    Returns
    -------
    str 
        DESCRIPTION.

    """
\end{minted}
\end{codebox}
\end{columns}
%
\end{frame}

% =========================================================================== %

\begin{frame}[fragile]
% : (int, float)
\begin{codebox}[Example: \texttt{concatFunc.py}]
\begin{minted}[linenos, fontsize=\scriptsize]{python3}
def concatenateNFold(f, x, n : int) :
    for i in range(n) :
        x = f(x)
    return x

f = lambda x : x**2
print( concatenateNFold(f, 2, 3) )
print( concatenateNFold(f, 2, 2.5) )
\end{minted}
\end{codebox}
%
\begin{cmdbox}[Output: \texttt{python3 -m mypy concatFunc.py}]
\begin{minted}[linenos, fontsize=\scriptsize]{text}
comments.py:8: error: Argument 3 to "concatenateNFold" has incompatible type "float";
  expected "int"
\end{minted}
\end{cmdbox}
%
\begin{hintbox}[Silent Bugs]
\small
Not all Type Errors only raise exceptions -- most of them are much less benign!
\end{hintbox}
%
\end{frame}

% =========================================================================== %

\begin{frame}[fragile]{Polymorphic Functions}
%
\begin{itemize}
\item Some function can take multiple input types
	\begin{itemize}
	\item Special type hint class: \texttt{typing.Union[type1, type2, ...]}
	\item Load with \inPy{from typing import Unioin}
	\end{itemize}
\item Some functions return tuples
	\begin{itemize}
	\item Type hint class \texttt{typing.Tuple}
	\item Load with \inPy{from typing import Tuple}
	\end{itemize}
\end{itemize}
%
\begin{tcbraster}[raster columns=2,
                  raster equal height,
                  nobeforeafter,
                  raster column skip=0.5cm]
\begin{codebox}[Example: \texttt{polymorphic.py}]
\begin{minted}[fontsize=\scriptsize, linenos]{python3}
from typing import Union, Tuple

def f (x : Union[list, tuple]) \
-> Tuple[int, int] :
    return len(x), max(x)

print(f( tuple((1, 2, 3)) ))
\end{minted}
\end{codebox}
%
\begin{cmdbox}[Output: \texttt{mypy polymorphic.py}]
\begin{minted}[fontsize=\scriptsize, linenos]{text}
Success: no issues found in 1 source file
\end{minted}
\end{cmdbox}
\end{tcbraster}
%
\end{frame}

% =========================================================================== %

\begin{frame}[fragile]{NumPy and mypy...}
%
\begin{itemize}
\item Precompiled types of NumPy not analyzed by mypy
\item NumPy version >= 1.21 has a plugin, but
\item Not yet available via pip
\item Workaround: tell mypy to ignore unknown types
\item File \texttt{mypy.ini} or \texttt{.mypy.ini} in CWD
\end{itemize}
%
\begin{cmdbox}[mypy.ini]
\begin{minted}[fontsize=\scriptsize]{text}
[mypy]
ignore_missing_imports = True
#plugins = numpy.typing.mypy_plugin     # if you get numpy version >= 1.21
\end{minted}
\end{cmdbox}
%
\begin{hintbox}[Current Numpy Version]
\begin{minted}[fontsize=\scriptsize]{text}
python3 -c "import numpy as np; print(np.__version__)"
\end{minted}
\end{hintbox}
%
\end{frame}

% =========================================================================== %

\begin{frame}{Explicit Tests}
%
\begin{itemize}
\item Module \texttt{doctest} (installed by default) can use DocStrings for tests
\item Ignores most lines
\item Exception: lines that begin with \texttt{>{}>{}>} or \texttt{...} will be treated like normal Python code
\item Subsequent lines: expected console output
\item If equal -- no output
\item Otherwise: error report
\item Multiple tests in sequence possible
\item Invoke by \texttt{doctest.testmod()}
\item Usually only if \inPy{__name__ == "__main__"}
\item Details: \url{https://docs.python.org/3/library/doctest.html}
\end{itemize}
%
\end{frame}

% =========================================================================== %

\begin{frame}[fragile]
%
\begin{tcbraster}[raster columns=2,
                  raster equal height,
                  nobeforeafter,
                  raster column skip=0.5cm]
\begin{codebox}[Example: DocTest]
\begin{minted}[linenos, fontsize=\scriptsize]{python3}
import doctest

def add(a : int, b : int) -> int :
    """ Return the sum of two integers
    Examples:
    >>> add(2, 3)
    5
    
    >>> x = 2
    >>> for y in  range(2, 4) :
    ...     add(x, y)
    4
    5
    
    >>> add(2, 2)
    5"""
    return a + b

if __name__ == "__main__" :
    print( doctest.testmod() )
\end{minted}
\end{codebox}
%
\begin{cmdbox}[Output: DocTest]
\begin{minted}[fontsize=\scriptsize, linenos]{text}
***********************************
File "doctest.py", line 15, in 
  __main__.add
Failed example:
    add(2, 2)
Expected:
    5
Got:
    4
***********************************
1 items had failures:
   1 of   4 in __main__.add
***Test Failed*** 1 failures.
TestResults(failed=1, attempted=4)
\end{minted}
\end{cmdbox}

%
\end{tcbraster}
%
\end{frame}

% =========================================================================== %

\begin{frame}{What to test}
%
\begin{tcolorbox}[title=Tweet by @brenankeller{,} Engineer at snapchat]
\small
A QA engineer walks into a bar. Orders a beer. Orders 0 beers. Orders 99999999999 beers. Orders a lizard. Orders -1 beers. Orders a ueicbksjdhd. 

First real customer walks in and asks where the bathroom is. The bar bursts into flames, killing everyone.
\begin{flushright}
\tiny
\url{https://twitter.com/brenankeller/status/1068615953989087232}
\end{flushright}
\end{tcolorbox}
%
There is no such thing as too much testing. But at the very least, you should test:
\begin{itemize}
\item Standard examples
\item Edge cases: large and small values, negative values, zero/null strings/empty lists/...
\item Different input types
\item Whether errors are really raised
\end{itemize}
%
\end{frame}

% =========================================================================== %

\begin{frame}[fragile]{More Extensive Tests}
%
\begin{itemize}
\item Tests in DocStrings \enquote{take up} space for documentation
\item Hard to put complex scenarios in there
\item Alternative: Module \texttt{unittest} (pre-installed)
\item Define \inPy{class}es of tests
	\begin{itemize}
	\item Derived of \texttt{unittest.TestCase}
	\item Each method is a test function
	\item Invoked by \texttt{unittest.main()}
	\item Fail/Pass based on several \emph{assert}s:
		\begin{itemize}
		\item \inPy{self.assertEqual(A, B)} -- pass if \texttt{A == B}
		\item \inPy{self.assertTrue(X)} -- pass if \texttt{X == True}
		\item \inPy{self.assertFalse(X)} -- guess what
		\item
\begin{minted}{python3}
with self.assertRaises(E) :
    codeThatRaisesE()
\end{minted}
		\end{itemize}
	\item Repetitive preparation code in method \inPy{setUp(self)}
	\end{itemize}
\item Details: \url{https://docs.python.org/3/library/unittest.html}
\end{itemize}
%
\end{frame}

% =========================================================================== %

\begin{frame}[fragile]
%
\begin{codebox}[Example: Unittests with unittest]
\begin{minted}[linenos, fontsize=\scriptsize]{python3}
import unittest

class TestStringMethods(unittest.TestCase):

    def test_upper(self):
        self.assertEqual('foo'.upper(), 'FOO')

    def test_isupper(self):
        self.assertTrue('FOO'.isupper())
        self.assertFalse('Foo'.isupper())

    def test_split(self):
        s = 'hello world'
        self.assertEqual(s.split(), ['hello', 'world'])
        # check that s.split fails when the separator is not a string
        with self.assertRaises(TypeError):
            s.split(2)

if __name__ == '__main__':
    unittest.main()
\end{minted}
\end{codebox}
%
\end{frame}

% =========================================================================== %

\begin{frame}
%
\begin{hintbox}[A special kind of thinking]
\begin{itemize}
\item When writing tests, try to \enquote{beat} yourself: find cases that have to break the code!
\item Remember: failed tests are a chance to improve your code. Undiscovered flaws are the true failres!
\item Dont't be afraid to test absurd cases -- they will be needed more often than you'd think
\item If possible, leave some time between writing the code and the test.
	If you don't have the code fresh in your memory, your mind is closer to the state when you're going to use the code and you'll more easily find the flaws!
\item Good unit tests are a sort of documentation, too!
\end{itemize}
\end{hintbox}
%
\end{frame}

% =========================================================================== %

\begin{frame}{Backups}
%
Github
%
\end{frame}